\begin{Slide}{Abstraction on the Goal-Design-scale}

\textit{why} $\rightarrow$ \textit{what} $\rightarrow$  \textit{how}:
\begin{itemize}
\item \textbf{Goal-level}: why? 
\begin{itemize}
\item focus on intentions of stakeholders and users
\end{itemize}
\item \textbf{Domain-level}: what do users do with the system?
\begin{itemize}
\item focus on usage context of a feature, normal and exceptional usage, domain events
\end{itemize}
\item \textbf{Product-level}: what does the system do?
\begin{itemize}
\item focus on system behavior, input-logic-state-output, normal and exceptional input/output, product events
\end{itemize}
\item \textbf{Design-level}: how? 
\begin{itemize}
\item up-front design choices, implementation details
\item really required/justified? often better as example only, not req

\end{itemize}
\end{itemize}
Which level is best? It depends. They are often combined.
\begin{itemize}
\item Too much 'how' may over-constrain the solution space giving too little freedom for developers to find the best solution.  
\item Without 'why' the risk of an unsuccessful solution is high.

\end{itemize}
\end{Slide}