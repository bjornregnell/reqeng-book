\begin{Slide}{Abstraction on the Goal-Design-scale}

\textit{why} $\rightarrow$ \textit{what} $\rightarrow$  \textit{how}:
\begin{itemize}
\item \textbf{Goal-level}: why? 
\begin{itemize}
\item intentions of stakeholders and users
\end{itemize}
\item \textbf{Domain-level}: what do users do with the system?
\begin{itemize}
\item focus on how users' tasks are supported by the system
\end{itemize}
\item \textbf{Product-level}: what does the system do?
\begin{itemize}
\item focus on system behavior in terms of input-logic-state-output
\end{itemize}
\item \textbf{Design-level}: how? 
\begin{itemize}
\item up-front design choices
\item are they really required and justified?  

\end{itemize}
\end{itemize}
Which level is best? It depends. They can be combined.
\begin{itemize}
\item Too much 'how' may over-constrain the solution space giving too little freedom for developers to find the best solution.  
\item Without 'why' the risk of an unsuccessful solution is high.

\end{itemize}
\end{Slide}