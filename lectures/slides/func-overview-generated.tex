\begin{Slide}{Examples of functional modeling techniques}
\begin{tabular}{l l  l}
\textbf{\emph{Context}} & \textbf{\emph{Data}}  & \textbf{\emph{Business Logic}}  \\ \hline
 Business model     &  Data dictionary     & User stories\\
 Stakeholder model  &  Data views          & Use cases and tasks \\
 Goal model         &  E/R-diagrams        & Narratives \\
 Context diagram    &  Class diagrams      & State diagrams \\
                    &  Regular expressions & Interaction diagrams\\
                    &  Protocol buffers    & Data-flow diagrams \\
\end{tabular}
%\vspace{0.5em}
%{\itshape\footnotesize Singularis: often only one, Pluralis: often many needed}

\begin{itemize}
\item \textit{Singularis:} often only one, \textit{pluralis:} often many needed %\vspace{0.5em}
\item Common complement to any technique:
\begin{itemize}
\item \textbf{Feature requirements} in natural language:
\begin{itemize}
\item explicit textual requirements ''The system shall...''
\item a product property that can be selected for a release or postponed
\item can combine functional aspects (data, logic) and quality aspects
\item link to above models


\end{itemize}
\end{itemize}
\end{itemize}
\end{Slide}