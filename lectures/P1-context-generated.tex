%!TEX encoding = UTF-8 Unicode
\documentclass{reqenglecture}

\title{Introduction to Software Requirements Engineering}

\subtitle{Part 1: Domain Knowledge}

\author{Björn Regnell}

\date{\vspace{1em}\footnotesize Updated: \today~
\\ License: CC-BY-SA 
\\ \url{https://github.com/bjornregnell/reqeng-book} 
}

%\beamerdefaultoverlayspecification{<+->} %de-comment if you want pause after items

\begin{document}
\maketitle

\begin{frame}
\frametitle{Part 1: Domain Knowledge}
\framesubtitle{Outline}
\tableofcontents
\end{frame}


\LectureOnly{\section{Introduction}}

\begin{Slide}{What is Requirements Engineering (RE)?}


\begin{minipage}[t]{0.65\textwidth}
\begin{itemize}
\item RE is focused on the
\begin{itemize}
\item \textbf{features} of software systems 
\item \textbf{system context}, including users and connected systems
\item \textbf{development context}, including stakeholders' intentions 
\end{itemize}
\end{itemize}
\end{minipage}%
\begin{minipage}[t]{0.35\textwidth}
\vspace{-1em}\hfill\includegraphics[width=0.8\textwidth]{../img/phone-support}
\end{minipage}%


\begin{itemize}
\item The RE process involves 
\begin{itemize}
\item knowledge-building \hfill research
\item consensus-building \hfill agree
\item decision-making    \hfill choose
\item innovation         \hfill generate ideas
\item communication      \hfill be pedagogical


\end{itemize}
\end{itemize}
\end{Slide}

\begin{Slide}{What is a requirement?}

\begin{minipage}[t]{0.65\textwidth}
\begin{itemize}
\item A simple definition:
\begin{itemize}
\item Something \textbf{needed} or \textbf{wanted}.
\item A documented \textbf{representation} of\\something needed or wanted.
\end{itemize}
\end{itemize}
\end{minipage}%
\begin{minipage}[t]{0.35\textwidth}
\vspace{-1em}\hfill\includegraphics[width=0.5\textwidth]{../img/light-bulb}
\end{minipage}%

\begin{itemize}
\item Are we representing what is \textbf{actually} needed or wanted? 

\item \textit{'Requirement'} can in practice mean many different things:\\
  must, option, idea, innovation, intent, rationale, function, quality, design, feature, decision, constraint, ...

\item The most \textbf{general} meaning:\\
  \textit{any} kind of \textbf{information entity} used in RE

\end{itemize}
\end{Slide}

\begin{Slide}{Core Activities of RE}

%{\vspace{1em}\resizebox{!}{3em}{\hspace{0.5em}{\bf ?}\hspace{0.5em}\WritingHand\hspace{0.5em}\Checkedbox\hspace{0.5em}\LeftScissors}\vspace{0.5em}}

\begin{itemize}
\item The 4 core activities of RE are: 
\begin{itemize}
\item \textbf{Elicitation} \hfill learning
\item \textbf{Specification} \hfill representing
\item \textbf{Validation}  \hfill checking
\item \textbf{Selection}   \hfill deciding
\end{itemize}
\item In practice, these activities are often
\begin{itemize}
\item \textbf{Interdependent} \hfill output of one is input to others
\item \textbf{Concurrent} \hfill one activity triggers others
\item \textbf{Continuous} \hfill throughout the product's life as it evolves

\end{itemize}
\end{itemize}
\end{Slide}

\begin{Slide}{What is good RE?}

\begin{itemize}
\item Feasible and helpful foundation for software development
\item Cost-effective process with high artifact quality
\item Happy stakeholders
\item Good system 
\begin{itemize}
\item commercially successful
\item beneficial to its users
\item ethical, helpful to society
\end{itemize}
\item When are we ready? What is good enough?

\end{itemize}
\end{Slide}

\begin{Slide}{RE in the Development Process}

\begin{itemize}
\item RE interprets stakeholders intentions into validated req specs
\item RE provides input to, and learns from down-stream activities
\begin{itemize}
\item System Design
\begin{itemize}
\item Quality reqs determine architectural decisions
\end{itemize}
\item System Implementation
\begin{itemize}
\item Functional reqs (data and logic) are realized in code  
\end{itemize}
\item System Verification 
\begin{itemize}
\item The req spec define correct output in test cases
\end{itemize}
\item System Operation
\begin{itemize}
\item User feedback is input to requirements evolution
\end{itemize}
\end{itemize}
\item As requirements evolve you must manage impact of changes
\item Traceability: 
\begin{itemize}
\item Links among artifacts to support change management
\item Forwards: from requirements to down-stream activities
\item Backwards: from requirements to stakeholders

\end{itemize}
\end{itemize}
\end{Slide}

\LectureOnly{\section{Purpose}}

\begin{Slide}{Requirements as Solution Constraints}

\begin{itemize}
\item U: the \textbf{universe} of all possible software systems

\item S: the \textbf{solution space}, a subset of U including\\all systems that \textbf{fulfill the spec}

\item S contains both ''\textbf{good}'' and ''\textbf{bad}'' systems

\item The \textbf{general purpose} of RE:
\begin{itemize}
\item to \textbf{constrain the solution space} so that software development is likely to produce a \textbf{good enough} solution

\end{itemize}
\item The reqt spec should be a good enough definition of what we mean with a ''good enough solution''

\item RE is the \textbf{foundation for software quality}.


\end{itemize}
\end{Slide}

\begin{Slide}{Common Acronyms}

\begin{itemize}
\item RE   \hfill requirements engineering
\item SE   \hfill software engineering
\item req  \hfill requirement 
\item spec \hfill specification
\item SRS  \hfill software (or system) requirements specification
\item sys  \hfill system
\item SW   \hfill software
\item dev  \hfill development
\item ops  \hfill operations
\item FR   \hfill quality requirements
\item QR   \hfill functional requirements



\end{itemize}
\LectureOnly{\section{Purpose}}
\end{Slide}


\begin{Slide}{What is a Requirements Specification?}

\begin{itemize}
\item A collection of requirements models with supporting information to help interpretation

\item Expressed in a combination of suitable styles:
\begin{itemize}
\item natural language
\item formal language (controlled syntax and semantics)
\item diagrams
\item tables
\item pictures
\item videos
\item prototypes
\item ...

\end{itemize}
\item Similar to a shopping list:
\begin{itemize}
\item You don't always get what you want.
\item You often want things that you don't need.

\end{itemize}
\end{itemize}
\end{Slide}

\begin{Slide}{Different kinds of requirements}

\begin{itemize}
\item Requirements are often labeled as:
\begin{itemize}
\item \textbf{Functional Requirements} (FR), including:
\begin{itemize}
\item Requirements on \textbf{Logic}
\item Requirements on \textbf{Data}
\end{itemize}
\item \textbf{Quality Requirements} (QR)
\begin{itemize}
\item Accuracy, Capacity, Performance, Reliability, Usability, Safety, Security, ...
\end{itemize}
\end{itemize}
\item In practice FR and QR are often combined and related:
\begin{itemize}
\item Functions have quality:
\begin{itemize}
\item a function can be unreliable due to bugs 
\end{itemize}
\item Logic and data is related: 
\begin{itemize}
\item functions have input, state, output
\end{itemize}
\item Quality is supported by functions: 
\begin{itemize}
\item a login function supports system security

\end{itemize}
\end{itemize}
\end{itemize}
\end{Slide}

\begin{Slide}{Requirements at different levels}

\begin{itemize}
\item Abstraction level: The goal-design scale
\item Levels of detail
\item \TODO{}

\end{itemize}
\end{Slide}
\begin{Slide}{Abstraction on the Goal-Design-scale}

From \textit{why} to \textit{how}:
\begin{itemize}
\item Goal-level: why? intentions of stakeholders and users
\item Domain-level: what users do? how users' tasks are supported by the system to achieve goal
\item Product-level: what the system does? system behavior in terms of input-logic-output
\item Design-level: how? up-front design choices; are they really required and justified?  

\end{itemize}
Which level is best? It depends.
\begin{itemize}
\item Too much 'how' may over-constrain the solution space giving too little freedom for developers to find the best solution.  
\item Without 'why' the risk is high of an unsuccessful solution.

\end{itemize}
\end{Slide}
\begin{Slide}{Levels of Formality}
From unstructured to mathematical:
\begin{itemize}
\item Very informal: free-form representation, no explicit rules
\item Very formal: syntax, semantics, inference, meta-language
\item Pro: Formality enables automatic checks, concise models, ...
\item Con: Formalization requires effort, knowledge, skills, ...




\end{itemize}
\end{Slide}
\begin{Slide}{Explicit or implicit requirements?}

\begin{itemize}
\item An explicit requirement: 
\begin{itemize}
\item has a unique id, such as a mnemonic (short name) or number
\item often has status, priority, or similar 
\item often has an explicit ''shall''-statement
\item often has links to other related explicit reqs
\end{itemize}
\item Implicit requirements:
\begin{itemize}
\item Parts of a spec but without unique id, status, ''shall'' 
\item Diagrams, general text: are they actual requirements or just help for the reader?
\end{itemize}
\item Advice: 
\begin{itemize}
\item Make at least the most important requirements explicit.
\item Combine diagrams with explanatory text; make explicit statements of what is a requirement or not in the diagram. 



\end{itemize}
\end{itemize}
\end{Slide}

\begin{Slide}{What is good Req Spec?}

\begin{itemize}
\item \TODO{}
\begin{itemize}
\item Correct \hfill represents the actual needs of stakeholders

\end{itemize}
\end{itemize}
\end{Slide}

\LectureOnly{\section{Context}}

\begin{Slide}{Development Context}

Almost everything related to RE depends the context!

\begin{itemize}
\item Stakeholder configuration: Relation Customer -- Supplier

\item Economy 
\begin{itemize}
\item Funding and resources
\item Business model

\end{itemize}
\item Examples of different kinds of projects:
\begin{itemize}
\item B2B: both customer and supplier is a company
\item B2C: the supplier provides a product to a consumer market
\item Public tender: a public authority is inviting companies to bid
\item In-house: a company develop a system for internal use
\item Open-source project

\end{itemize}
\item Some questions to consider:
\begin{itemize}
\item Who has the knowledge?
\item Who has the decision power?
\item Who gets the biggest value/profit?
\begin{itemize}
\item Short-term versus Long-term
\end{itemize}
\item Who takes the biggest risk?

\end{itemize}
\end{itemize}
\end{Slide}
\begin{Slide}{Type of product}
\begin{itemize}
\item Level of customization
\begin{itemize}
\item generic
\item customer specific
\end{itemize}
\item Hardware integration:
\begin{itemize}
\item HW+SW 
\item Pure SW
\end{itemize}
\item Network integration
\begin{itemize}
\item off-grid
\item connected
\item distributed
\item concurrent massive multi-user online communication, ...

\end{itemize}
\end{itemize}
\end{Slide}
\begin{Slide}{Examples of common RE Contexts:}
\begin{itemize}
\item Public tender: a public authority invites suppliers to bid
\item B2B: both customer and supplier are companies
\item B2C: the supplier provides SW to a consumer market
\item In-house: one org develops system for internal use
\item Open-source library: organisations share SW investments 
\item Embedded system
\item Webb app: backend-frontend
\item High-assurance systems: security and safety is critical


\end{itemize}
\end{Slide}
\begin{Slide}{Scale}

RE challenges increase with scale!

\TODO{}
\begin{itemize}
\item Small-scale RE
\item Medium-scale RE
\item Large-scale RE
\item Very large-scale RE



\end{itemize}
\end{Slide}
\begin{Slide}{Context Diagram}

\begin{itemize}
\item A diagram describing the environment of the product
\item The named product in the center as a \textbf{closed} box
\begin{itemize}
\item no internal structure is shown -- the focus is on context
\item open box with system parts inside is an \textit{architecture} diagram

\end{itemize}
\item Entities interacting with the product are connected by arrowed lines to show data flow direction
\begin{itemize}
\item User roles (actors), shown as straw man icons
\item Other connected systems, shown as named closed boxes
\end{itemize}
\item \textbf{Inner domain}: \textit{direct} interaction with product
\item \textbf{Outer domain}: \textit{indirect} interaction with product
\begin{itemize}
\item often \textit{not} included in the context diagram
\end{itemize}
\item Accompanying explaining text, including explicit requirements: ''the system shall have interface X''


\end{itemize}
\end{Slide}
\LectureOnly{\section{Elicitation}}
\begin{Slide}{Elicitation}


\begin{itemize}
\item \TODO{}

\end{itemize}
\end{Slide}

\LectureOnly{\section{Prioritization}}

\begin{Slide}{Prioritization}
Hello
\end{Slide}
* \TODO{}
\end{document}