\documentclass{reqengbook}

\title{
\bf\sffamily\fontsize{22}{28}\selectfont
Requirements Engineering for Software Developers
}

\author{\sffamily\fontsize{20}{30}\selectfont Björn Regnell}

\date{\vspace{2em}\sffamily\small Updated: \today 
\\ License: CC-BY-SA 
\\ \url{https://github.com/bjornregnell/reqeng-book} 
}

\begin{document}
%\pagenumbering{roman} 

% \begin{minipage}{0.5\textwidth}
%   \maketitle%
% \end{minipage}%

\hspace{-15mm}\begin{minipage}{1.0\textwidth}
\centering\vspace{-15mm}%
\begin{tikzpicture}
  \draw (9, -8) node  {\includegraphics[width=0.65\textwidth]{../img/phone-support.jpg}};
  \draw (0, 0) node[text width=0.5\textwidth] {\maketitle};
  \draw (-1, -8) node [text width=0.25\textwidth] {\small\textit{Contributors:}\\Your name here?\\contributions are welcome, contact \texttt{bjorn.regnell@cs.lth.se}};
\end{tikzpicture}

\end{minipage}%

% \begin{minipage}{0.5\textwidth}
%   \centering\vspace{2.2cm}
%   \includegraphics[width=0.9\textwidth]{../img/phone-support.jpg}%
% \end{minipage}%
\pagebreak

\setcounter{tocdepth}{3}


\begin{multicols*}{3}  % * means unbalanced columns allowed
\fontsize{8.4}{10}\selectfont
%\fontsize{9}{11}\selectfont

\tableofcontents
\end{multicols*}


\setcounter{chapter}{-1}
\begin{multicols*}{2}  % * means unbalanced columns allowed
\chapter{Preface}
About the book. Audience. How to read. How to use in a course, by students and teachers.
\end{multicols*}


\newgeometry{hmargin={45mm,148.5mm},vmargin={15mm,15mm}}
\pagebreak


\part{Domain Knowledge}  %%%%%%%%%%%%%%%%%%%%%%%%%%%%

\input{../chapters/01-intro.tex}
\input{../chapters/02-context.tex}
%!TEX encoding = UTF-8 Unicode
%!TEX root = ../book/reqeng-book.tex

\chapter{Elicitation}%
blabla


Elicitation Goals \& Challenges%

\section{What information to elicit?}

\begin{itemize}
  \item Present
  \item Future 
  \item Consensus 
  \item Commitment 
  \item Innovative ideas
  \item Goals -- Domain -- Product -- Design
  \item Priorities (cost, benefit, risk of requriements)
\end{itemize}

\section{Why is elicitation so hard?}

Elicitation challenges (barriers/difficulties) 

\section{Engaging with stakeholders}

\subsection{Surveys}

\subsection{Interviews}

\begin{itemize}
\item Unstructured interviews: open questions, open topics
\item Structured interviews: closed questions, focused topics
\item Semi-structured: combine both
\end{itemize}


\subsection{Case studies}

Demonstrations

Observation of Current work 

Prototyping -> its own chapter 

Usability testing -> own subchapter in Module on Quality

Pilot system deployment

\subsection{Creativity methods}

facilitate stakeholders in idea generation 

assess novelty and market 

brainstorming

focus-groups

storyboarding

From \url{https://en.wikipedia.org/wiki/Storyboard}:
''Storyboarding is used in software development as part of identifying the specifications for a particular set of software. During the specification phase, screens that the software will display are drawn, either on paper or using other specialized software, to illustrate the important steps of the user experience. The storyboard is then modified by the engineers and the client while they decide on their specific needs. The reason why storyboarding is useful during software engineering is that it helps the user understand exactly how the software will work, much better than an abstract description. It is also cheaper to make changes to a storyboard than an implemented piece of software. ''



\section{Elicitation in product operation}

\subsection{User engagement}
support issues, user forums and social media

\subsection{Telemetry and A/B-testing}

\subsection{Business Intelligence}
(på svenska: omvärldsanalys)

study similar companies, competitors, subcontractors, suppliers 


\input{../chapters/04-prio.tex}

\part{Functionality}  %%%%%%%%%%%%%%%%%%%%%%%%%%%%


%!TEX encoding = UTF-8 Unicode
%!TEX root = ../book/reqeng-book.tex

\chapter{Data}%

\section{What is Data?}

Functional Requirements on Data, or shorter Data Requirements (DR). What the system stores. Domain information. Input and output data formats.

\url{https://en.wikipedia.org/wiki/Information_model}

\url{https://en.wikipedia.org/wiki/Data_modeling}

Domain-specific example:\\\url{https://en.wikipedia.org/wiki/Building_information_modeling}


\section{Data dictionaries}
Add text here. Add more text here.

\section{Data views}
Add text here. Add more text here.  (aka Virtual Windows)

\section{Data diagrams}

E/R-diagrams came first, connected to database design.

Class diagrams part of UML, similar to E/R-diagrams but include inheritance.

\subsection{Entity-Relationship diagrams}
Add text here. Add more text here.

\subsection{Class diagrams}
Add text here. Add more text here.


%\section{Formal Data Specification}%

\section{Data format specification}

Formal data format specification methods

\subsection{Regular expressions (regexp)}


\url{https://en.wikipedia.org/wiki/Regular_expression}


\subsection{Protocol buffers (protobuf)}

Protocol data specification

\url{https://en.wikipedia.org/wiki/Protocol_Buffers}

%!TEX encoding = UTF-8 Unicode
%!TEX root = ../book/reqeng-book.tex

\chapter{Logic}%

Functional Requirements on business logic that provide constraints on system behavior.


\section{User-system work split}

\section{Contextual usage modeling}%

\subsection{User stories}

A user story is an informal, general explanation of a software feature written from the perspective of the end user. Its purpose is to articulate how a software feature will provide value to the customer.

\url{https://en.wikipedia.org/wiki/User_story}

\url{https://www.atlassian.com/agile/project-management/user-stories}

\subsection{Use cases and tasks}

\subsection{Narratives and personas}

Narrative: Rich and detailed description of a real-world usage situation in the form of a short-story.

Persona: a fictional character representing an imagined human user, often attributed with concrete personal attitudes and desires. 


\section{System behavior modeling}

-- Domain Events

-- Product Events

-- Product Functions triggered by Events: (Input, State) => Output

\subsection{State diagram}%
\url{https://en.wikipedia.org/wiki/State_diagram} 

\subsection{Interaction diagrams}

\begin{itemize}
  \item UML Sequence Diagrams \url{https://en.wikipedia.org/wiki/Sequence_diagram}
  \item ITU Message Sequence Charts \url{https://en.wikipedia.org/wiki/Message_sequence_chart}
\end{itemize}

\subsection{Data-flow diagrams}%
\url{https://en.wikipedia.org/wiki/Data-flow_diagram} 





\chapter{Prototyping}

\section{Why prototype?}

What is it?

Can support elicitation, but also specification, validation and selection. Explain how...

Why do it? Purpose of prototyping.

\begin{itemize}
  \item Exploration and learning
  \item Communication: sales, aligment
  \item Incremental development
  \item Quality improvement
  \item Validation and testing
  \item Problem-solution fit, product-market fit, technical feasibility, usability testing
\end{itemize}

\section{Prototype scoping}

\begin{itemize}
  \item Breadth of functionality
  \item Functional refinement
  \item Visual appearance
  \item Interactive and haptic behavior
  \item Data realism
\end{itemize}


\section{Prototype media}
paper or computer-based
\begin{itemize}
  \item Sketch 
  \item Wireframe
  \item Mockup
  \item Source-code software
  \item Other: video, interview
\end{itemize}

\section{Prototype usage}
\begin{itemize}
  \item Reviewers: internal, FFF, external
  \item Prototype interaction: yes no (demo)
  \item Review approach: scenario-based, free
  \item Usage environment
\end{itemize}

\section{Exploration strategy}
\begin{itemize}
  \item Single vs parallell Exploration
  \item Iteration focus: business, product, feature, optimization
  \item Iteration size
\end{itemize}


\chapter{Delegated Requirements}

\section{Standards as requirements}

\section{Regulatory requirements}

\subsection{General Data Protection Regulation} 

\subsection{Artificial Intelligence Act}
Point to chapter on AI-act in "Special Domains"

\section{Test cases as requirements}

\section{Development process requirements}


\part{Quality} %%%%%%%%%%%%%%%%%%%%%%%%%%%%

%%%%%%%%%%%%%%%%%%%%%%%%%%%%%%%%%%%%%%%%%%%

\chapter{Quality Requirements}
\section{What is product quality?}
Different Quality Models. Relation between product-level quality and domain-level quality.

\url{https://en.wikipedia.org/wiki/Software_quality}

\url{https://en.wikipedia.org/wiki/ISO/IEC_9126}

\section{Product-level qualities}
\subsection{Accuracy}

Accuracy of data representation. Relates to Data model and  representation of data fields in records.

Reqs on: How to handle rounding errors?

\subsection{Capacity}
Subcategory of efficiency.

\subsection{Performance}
Subcategory of efficiency.

\subsection{Reliability}

\subsection{Maintainability}

\section{Domain-level qualities}

\subsection{Usability and UX}

Interaction requirements, quality of interaction design.

\subsection{Security and safety}

\url{https://en.wikipedia.org/wiki/Information_security}

\section{Quality road-mapping with QUPER}





%%%%%%%%%%%%%%%%%%%%%%%%%%%%%%%%%%%%%%%%%%%%%%%%%%
\chapter{Requirements Validation}

\section{Specification quality aspects}

\TODO{decide how much here and how much in intro on spec quality}

\section{Requirements inspections}
static validation of req models

capture-recapture

fault seeding

\section{Dynamic validation}

Dynamic validation methods:
\begin{itemize}
  \item Usability testing
  \item Prototyping
  \item Simulation
\end{itemize}

\subsection{Usability testing}

\subsection{Simulation}

\TODO{Or should Similuation get it's on chapter after Prototyping in Part II Functionality???}

\chapter{Product scoping}

\section{Release themes and epics}

bundles of features and use cases, a bundle that delivers value

\section{Requirements status ladder}

\section{Release planning} 

\chapter{Product verification}
\section{Unit testing}
\section{System testing}
\section{Acceptance testing}
\section{Regression testing}

\part{Special Contexts}

\chapter{Special Process Modes}

\section{Agile RE}

\section{Open Source RE}

\section{Continuous RE}

\subsection{CI/CD}

\subsection{DevOps}

\section{Contract-based RE} 

\subsection{Subcontracting and integration}
\subsection{Public procurement}

\section{RE in product-line engineering}
\subsection{Product-lines and product families}
\subsection{Variability modeling}

\chapter{Special Product Types}

\section{RE for High-Assurance Systems}
\subsection{Information security audit}
\url{https://en.wikipedia.org/wiki/Information_security_audit}

\section{RE for Artificial Intelligens}

\section{RE for Embedded Systems}
\subsection{Hardware-Software co-design}

\part{Appendices}\appendix

\chapter{Getting Started with \texttt{reqT}}

\chapter{Tool Lab: Prioritization}

\chapter{Tool Lab: Release Planning}

\chapter{Course Project}

\section{Literature Studies}

\section{System Ideas}

\section{Course Project Outline}
\subsection{Management roles}
\subsection{Release 1: Scoped}
\subsection{Release 2: Specified}
\subsection{Release 3: Validated}

\newpage
\subsection{Assessment matrix}

\begin{adjustbox}{width=2.0\textwidth}
\input{../chapters/project-assessment-matrix.tex}
\end{adjustbox} 

\chapter{Specification Templates}

\section{How to structure a linear document?}

\section{IEEE 830 Specification template}
\begin{mdframed}[backgroundcolor=green!8]
\begin{description}
\item [1.]  \textbf{Introduction}
\item [1.1] Purpose
\item [1.2] Scope
\item [1.3] Definitions, acronyms, and abbreviations
\item [1.4] References
\item [1.5] Overview
\item [2.]  \textbf{Overall description}
\item [2.1] Product perspective
\item [2.2] Product functions
\item [2.3] User characteristics
\item [2.4] Constraints
\item [2.5] Assumptions and dependencies
\item [3.]       \textbf{Specific requirements}
\item [3.1]      External interface requirements
\item [3.1.1]    User interfaces
\item [3.1.2]    Hardware interfaces
\item [3.1.3]    Software interfaces
\item [3.1.4]    Communications interfaces
\item [3.2]      Functional requirements
\item [3.2.1]    User class 1
\item [3.2.1.1]  Feature 1.1
\item [3.2.1.1.1] Introduction/Purpose of feature
\item [3.2.1.1.2] Stimulus/Response sequence
\item [3.2.1.1.3] Associated functional requirements
\item [3.2.1.2]   Feature 1.2
\item [3.2.1.2.1] Introduction/Purpose of feature
\item [3.2.1.2.2] Stimulus/Response sequence
\item [3.2.1.2.3] Associated functional requirements
\item [...]
\item [3.2.1.m] Feature 1.m
\item [3.2.1.m.1] Introduction/Purpose of feature
\item [3.2.1.m.2] Stimulus/Response sequence
\item [3.2.1.m.3] Associated functional requirements
\item [3.2.2] User class 2
\item [...]
\item [3.2.n] User class n
\item [...]
\item [3.3] Performance requirements
\item [3.4] Design constraints
\item [3.5] Software system attributes
\item [3.6] Other requirements
\item \textbf{Appendixes}
\item \textbf{Index}
\end{description}
\end{mdframed}

\chapter{Learning from RE Disasters}

\section{Swedish National Police System}

\url{https://computersweden.se/article/1304245/haveriet-inifran-sa-gick-pust-fran-succe-till-fiasko.html}

\section{Swedish Regional Health Care System}

\url{https://computersweden.se/article/1276485/miljardprojekt-i-varden-underskattade-lagstiftning-och-forsenades-rejalt.html}

\url{https://lakartidningen.se/aktuellt/nyheter/2022/04/skane-staller-krav-pa-cerner-om-systemet-millennium/}

\url{https://lakartidningen.se/aktuellt/nyheter/2022/02/nytt-journalsystem-drar-ut-pa-tiden-vgr-vill-ha-pengar-tillbaka/}

\url{https://www.svt.se/nyheter/lokalt/vast/brister-i-nya-patientsystemet-har-kostat-skattebetalarna-en-miljard}

\end{document}
