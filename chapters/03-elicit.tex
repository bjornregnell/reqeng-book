%!TEX encoding = UTF-8 Unicode
%!TEX root = ../book/reqeng-book.tex

\chapter{Elicitation}%
blabla


Elicitation Goals \& Challenges%

\section{What information to elicit?}

\begin{itemize}
  \item Present
  \item Future 
  \item Consensus 
  \item Commitment 
  \item Innovative ideas
  \item Goals -- Domain -- Product -- Design
  \item Priorities (cost, benefit, risk of requriements)
\end{itemize}

\section{Why is elicitation so hard?}

Elicitation challenges (barriers/difficulties) 

\section{Engaging with stakeholders}

\subsection{Surveys}

\subsection{Interviews}

\begin{itemize}
\item Unstructured interviews: open questions, open topics
\item Structured interviews: closed questions, focused topics
\item Semi-structured: combine both
\end{itemize}


\subsection{Case studies}

Demonstrations

Observation of Current work 

Prototyping -> its own chapter 

Usability testing -> own subchapter in Module on Quality

Pilot system deployment

\subsection{Creativity methods}

facilitate stakeholders in idea generation 

assess novelty and market 

brainstorming

focus-groups

storyboarding

From \url{https://en.wikipedia.org/wiki/Storyboard}:
''Storyboarding is used in software development as part of identifying the specifications for a particular set of software. During the specification phase, screens that the software will display are drawn, either on paper or using other specialized software, to illustrate the important steps of the user experience. The storyboard is then modified by the engineers and the client while they decide on their specific needs. The reason why storyboarding is useful during software engineering is that it helps the user understand exactly how the software will work, much better than an abstract description. It is also cheaper to make changes to a storyboard than an implemented piece of software. ''



\section{Elicitation in product operation}

\subsection{User engagement}
support issues, user forums and social media

\subsection{Telemetry and A/B-testing}

\subsection{Business Intelligence}
(på svenska: omvärldsanalys)

study similar companies, competitors, subcontractors, suppliers 

