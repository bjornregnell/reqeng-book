%!TEX encoding = UTF-8 Unicode
%!TEX root = ../book/reqeng-book.tex

\chapter{Prioritization}%
blabla

\section{Why prioritize?}%

\section{The prioritization process}

\subsection{Activities}
who does what when 

\subsection{Challenges}

Finding a good abstraction level

Combinatorial explosion

Inter-dependencies

Not easy to predict the future

Power and politics

\subsection{Aspects}

Example aspects:

-- Importance (e.g. financial benefit, urgency, strategic value, market share...)

--  Penalty (e.g. bad-will if requirement not included)

--  Cost (e.g., staff effort)

--  Time (e.g., lead time)

--  Risk (e.g., technical risk, business risk)

--  Volatility (e.g. scope instability, probability of change)

-- Other aspects: competitors, brand fitness, competence, release theme...

-- Combination of aspects: cost vs. benefit, cost vs. risk, importance vs. volatility

-- Optimize: minimize or maximize some combinations, e.g., cost vs. benefit

\section{Scales}

\section{Methods}

\textbullet Grouping, Numerical assignment (grading)

-- Can be done using any scale (categorical, ordinal, ratio)

-- Quick and easy; but a risk is that all reqs are deemed highly important as they are
not challenged against each other; may be misinterpreted as ratio scale (even if
”4” not necessarily is "twice as much" as ”2” when using an ordinal scale).

\textbullet 100-dollar-test

-- Ratio scale, quick and easy, risk of shrewd tactics (listigt taktikspel)

\textbullet Ranking (sorting)

-- Ordinal scale, pairwise comparison, easy and rather quick

\textbullet Top-ten (or Top-n)

-- Ordinal scale if the top list is ranked / Categorical grouping if not ranked; very
quick and simple, gives a rough estimate on a limited set of req

\textbullet (Analytical Hierarchy Process (AHP))

-- Ratio scale, pairwise comparison, tool is needed, redundancy gives measure of
consistency 
