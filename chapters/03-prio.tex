%!TEX encoding = UTF-8 Unicode
%!TEX root = ../book/reqeng-book.tex

\chapter{Prioritization}%
Relate to estimitation, prediction, KPI, metrics.

\TODO{should GQM be outlined here or under quality metrics; make ref to QR chapter}

\section{Why prioritize?}%

\section{Prioritization activities}
the prioritization process: who does what when 

\section{Prioritization challenges}
\begin{itemize}
  \item Finding a good abstraction level
  \item  Combinatorial explosion
  \item  Inter-dependencies
  \item   Not easy to predict the future
  \item  Power and politics
\end{itemize}


\section{Priority metrics}
\subsection{What aspects to prioritize?}

\TODO{Decide if use ''aspect'' or ''criteria''}

Example aspects:

-- Importance (e.g. financial benefit, urgency, strategic value, market share...)

--  Penalty (e.g. bad-will if requirement not included)

--  Cost (e.g., staff effort)

--  Time (e.g., lead time)

--  Risk (e.g., technical risk, business risk) Risk = damage * probability


--  Volatility (e.g. scope instability, probability of change)

-- Other aspects: competitors, brand fitness, competence, release theme...

-- Combination of aspects: cost vs. benefit, cost vs. risk, importance vs. volatility

-- Optimize: minimize or maximize some combinations, e.g., cost vs. benefit

\subsection{Priority scales}

\section{Prioritization methods}

\subsection{Categorizing}

Grouping, Numerical assignment (grading)

-- Can be done using any scale (categorical, ordinal, ratio)

-- Quick and easy; but a risk is that all reqs are deemed highly important as they are
not challenged against each other; may be misinterpreted as ratio scale (even if
”4” not necessarily is "twice as much" as ”2” when using an ordinal scale).

\subsection{Ordering}

Ranking, sorting

-- Ordinal scale, pairwise comparison, easy and rather quick


\subsection{Ratio-scale estimation}
  
\subsubsection{100-dollar-test}

-- Ratio scale, quick and easy, risk of shrewd tactics (listigt taktikspel)


\subsubsection{Top-ten (or Top-n)}

-- Ordinal scale if the top list is ranked / Categorical grouping if not ranked; very
quick and simple, gives a rough estimate on a limited set of req

\subsubsection{Analytical Hierarchy Process (AHP)}

-- Ratio scale, pairwise comparison, tool is needed, redundancy gives measure of
consistency 

