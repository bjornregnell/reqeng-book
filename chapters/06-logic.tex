%!TEX encoding = UTF-8 Unicode
%!TEX root = ../book/reqeng-book.tex

\chapter{Logic}%

Functional Requirements on business logic that provide constraints on system behavior.


\section{When to decide user-system work split?}

\section{Contextual usage modeling}%

\subsection{User stories}

A user story is an informal, general explanation of a software feature written from the perspective of the end user. Its purpose is to articulate how a software feature will provide value to the customer.

\url{https://en.wikipedia.org/wiki/User_story}

\url{https://www.atlassian.com/agile/project-management/user-stories}

\subsection{Use cases and tasks}

\subsection{Narratives and personas}

Narrative: Rich and detailed description of a real-world usage situation in the form of a short-story.

Persona: a fictional character representing an imagined human user, often attributed with concrete personal attitudes and desires. 


\section{System behavior modeling}

-- Domain Events

-- Product Events

-- Product Functions triggered by Events: (Input, State) => Output

\subsection{State diagram}%
\url{https://en.wikipedia.org/wiki/State_diagram} 

\subsection{Interaction diagrams}

\begin{itemize}
  \item UML Sequence Diagrams \url{https://en.wikipedia.org/wiki/Sequence_diagram}
  \item ITU Message Sequence Charts \url{https://en.wikipedia.org/wiki/Message_sequence_chart}
\end{itemize}

\subsection{Data-flow diagrams}%
\url{https://en.wikipedia.org/wiki/Data-flow_diagram} 


